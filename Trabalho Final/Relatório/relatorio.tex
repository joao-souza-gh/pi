\documentclass[12pt,a4paper]{article}

\usepackage[utf8]{inputenc}
\usepackage[T1]{fontenc}
\usepackage[brazil]{babel}
\usepackage{graphicx}
\usepackage{amsmath, amssymb}
\usepackage{booktabs}
\usepackage{float}
\usepackage{listings}
\usepackage{color}
\usepackage{cite}
\usepackage{geometry}
\geometry{a4paper, margin=2.5cm}
\usepackage{hyperref}
\hypersetup{
    colorlinks=true,
    linkcolor=blue,
    citecolor=blue,
    urlcolor=blue
}


\lstset{
    basicstyle=\footnotesize\ttfamily,
    breaklines=true,
    frame=single,
    backgroundcolor=\color{gray!5},
    captionpos=b
}

\title{Relatório de Desenvolvimento de Modelo de Classificação de Imagens com Redes Neurais Convolucionais}
\date{\today}

\begin{document}

\maketitle

\begin{flushright}
    Alexandre\footnote{Graduando em Ciência da Computação -- FURB.} \\
    Gustavo Guerreiro\footnote{Graduando em Ciência da Computação -- FURB.} \\
    João Martinho Schneider da Silva e Souza\footnote{Graduando em Ciência da Computação -- FURB.}
\end{flushright}



\textbf{Palavras-chave:} Redes neurais convolucionais, aprendizado profundo, classificação, processamento de imagem.

\section{Descrição do Problema}
Lorem ipsum dolor sit amet, consectetur adipiscing elit. Fusce non augue a odio tincidunt porta. 
Este trabalho tem como objetivo desenvolver um modelo de rede neural convolucional capaz de classificar imagens em diferentes categorias, a partir de uma base de dados pública.

\section{Montagem e Preparação da Base de Dados}
Lorem ipsum dolor sit amet, consectetur adipiscing elit. A base de dados foi composta por imagens obtidas de fontes abertas. 
Antes do treinamento, as imagens foram redimensionadas, normalizadas e divididas em conjuntos de treino, validação e teste.

\subsection{Pré-processamento}
\begin{itemize}
    \item Redimensionamento;
    \item Normalização;
    \item Aumento de dados;
    \item Separação 80\% para treino e 20\% para teste.
\end{itemize}

\section{Modelo / Arquitetura da Rede}
A arquitetura da CNN foi definida conforme segue:


\section{Treinamento, Classificação e Testes do Modelo}
Lorem ipsum dolor sit amet, consectetur adipiscing elit. 
O modelo foi treinado por 25 épocas com \textit{batch size} de 32, utilizando o otimizador Adam e função de perda categórica cruzada. 
A acurácia foi avaliada no conjunto de teste após cada época.

\section{Código-Fonte e Explicações}
A seguir, um trecho simplificado do código desenvolvido:



\section{Demonstração das Entradas e Saídas}
Lorem ipsum dolor sit amet, consectetur adipiscing elit. 
A Figura \ref{fig:exemplo} ilustra um exemplo de imagem de entrada e a respectiva previsão gerada pelo modelo.



\section{Apresentação e Discussão dos Resultados}
Lorem ipsum dolor sit amet, consectetur adipiscing elit. 
O modelo foi treinado por 25 épocas com \textit{batch size} de 32, utilizando o otimizador Adam e função de perda categórica cruzada. 
A acurácia foi avaliada no conjunto de teste após cada época.

\begin{table}[H]
\centering
\caption{Desempenho do modelo nos dados de teste.}
\label{tab:resultados}
\begin{tabular}{lcc}
\toprule
Métrica & Valor (\%) \\
\midrule
Acurácia & XX.X \\
Precisão & XX.X \\
Recall & XX.X \\
F1-score & XX.X \\
\bottomrule
\end{tabular}
\end{table}

\section{Conclusão}
Lorem ipsum dolor sit amet, consectetur adipiscing elit. 
O modelo proposto demonstrou resultados satisfatórios, podendo ser aprimorado com arquiteturas mais profundas ou técnicas de regularização mais avançadas.

\bibliographystyle{plain}
\begin{thebibliography}{9}

\bibitem{exemplo2025}
Sobrenome, Nome. (2025). \textit{Título do Artigo}. IEEE.

\end{thebibliography}

\end{document}
