\documentclass[12pt,a4paper]{article}

% -----------------------
% Pacotes básicos
% -----------------------
\usepackage[utf8]{inputenc}
\usepackage[T1]{fontenc}
\usepackage[brazil]{babel}
\usepackage{graphicx}
\usepackage{amsmath, amssymb}
\usepackage{booktabs}
\usepackage{float}
\usepackage{cite}
\usepackage{geometry}
\geometry{a4paper, margin=2.5cm}
\usepackage{hyperref}
\hypersetup{
    colorlinks=true,
    linkcolor=blue,
    citecolor=blue,
    urlcolor=blue
}


\title{Análise de Imagens de Ressonância Magnética com Redes Neurais Convolucionais para Apoio ao Diagnóstico da Doença de Alzheimer}
\author{
    Gustavo Guerreiro \\
    \textit{Curso de Ciência da Computação} \\
    \textit{FURB} \\
    \texttt{gustavo.guerreiro@furb.br}
}
\date{\today}

\begin{document}

\maketitle

\begin{abstract}
A Doença de Alzheimer é uma condição lorem.
\end{abstract}

\textbf{Palavras-chave:} Redes neurais convolucionais, Alzheimer, Ressonância magnética, Deep learning, Diagnóstico assistido por computador.

\section{Introdução}
lorem

\section{Trabalhos Relacionados}
Trabalhos relacionados

\section{Materiais e Métodos}

\subsection{Base de Dados}
Foi utilizado o dataset xyz.

\subsection{Pré-processamento}
O pré-processamento incluiu:
\begin{itemize}
    \item passo 1;
    \item passo 2;
    \item passo 3.
\end{itemize}

\subsection{Arquitetura da Rede}
A CNN implementada possui a seguinte estrutura:
\begin{itemize}
    \item passo 1;
    \item passo 2;
    \item passo 3;
    \item passo 4;
\end{itemize}

O treinamento foi realizado utilizando o otimizador Adam, taxa de aprendizado de 0.001 e função de perda \textit{categorical crossentropy}.

\subsection{Ambiente de Execução}
Os experimentos foram realizados em Python 3.10, utilizando TensorFlow e Keras, em GPU NVIDIA RTX 3060 com 12GB de VRAM.

% -----------------------
\section{Resultados}
A CNN atingiu uma acurácia média de 92\% no conjunto de teste, com uma matriz de confusão indicando bom equilíbrio entre sensibilidade e especificidade. A Figura~\ref{fig:roc} apresenta a curva ROC obtida.

\begin{figure}[H]
\centering
\includegraphics[width=0.6\textwidth]{roc_curve.png}
\caption{Curva ROC da CNN aplicada ao conjunto de teste.}
\label{fig:roc}
\end{figure}

\begin{table}[H]
\centering
\caption{Desempenho do modelo nos dados de teste.}
\begin{tabular}{lccc}
\toprule
Métrica & Valor (\%) \\
\midrule
Acurácia & 92.1 \\
Precisão & 90.5 \\
Sensibilidade & 93.2 \\
Especificidade & 91.0 \\
\bottomrule
\end{tabular}
\end{table}

\section{Discussão}
lorem

\section{Conclusão}
lorem
\section*{Referências}
\bibliographystyle{plain}
\begin{thebibliography}{9}

\bibitem{teste2025}
Sobrenome, Nome. \& Wei, Y. (2019). Teste de citação. \textit{IEEE}, 7, 166844–166852.

\end{thebibliography}

\end{document}
